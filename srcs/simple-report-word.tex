% 日本語で文書を作る設定
% タイトル・概要ページ付き
\documentclass[titlepage]{jlreq}

% フォントを指定
\usepackage{luatexja-fontspec}
\setmainfont[Ligatures={Rare, TeX}]{Times-New-Roman}
\setsansfont[Ligatures={Rare, TeX}]{Times-New-Roman}
\setmainjfont[
    YokoFeatures       = {JFM=jlreq}, 
    TateFeatures       = {JFM=jlreqv}, 
    BoldFont           = MS-Mincho, 
    BoldFeatures       = {FakeBold=2}, 
    ItalicFont         = MS-Mincho, 
    ItalicFeatures     = {FakeSlant=0.33}, 
    BoldItalicFont     = MS-Mincho, 
    BoldItalicFeatures = {FakeBold=2, FakeSlant=0.33}
]{MS-Mincho}
\setsansjfont[
    YokoFeatures       = {JFM=jlreq}, 
    TateFeatures       = {JFM=jlreqv}, 
    BoldFont           = MS-Mincho, 
    BoldFeatures       = {FakeBold=2}, 
    ItalicFont         = MS-Mincho, 
    ItalicFeatures     = {FakeSlant=0.33}, 
    BoldItalicFont     = MS-Mincho, 
    BoldItalicFeatures = {FakeBold=2, FakeSlant=0.33}
]{MS-Mincho}

\newjfontfamily\fontcap[
    YokoFeatures       = {JFM=jlreq}, 
    TateFeatures       = {JFM=jlreqv}, 
    BoldFont           = MS-Gothic, 
]{MS-Gothic}

% PCで見るように自動リンク
\usepackage{hyperref}

% hyperrefを使ってrefの種類を自動判別して接頭辞をつける
\def\equationautorefname~#1\null{\textrm{~(#1)式\;}\null}
\def\figureautorefname~#1\null{ 図~#1\null}
\def\tableautorefname~#1\null{ 表~#1\null}
\def\sectionautorefname~#1\null{第~#1章\null}
\def\subsectionautorefname~#1\null{~#1節\null}
\def\subsubsectionautorefname~#1\null{~#1節\null}
\def\paragraphautorefname~#1\null{第~#1パラグラフ\null}
\def\subparagraphautorefname~#1\null{第~#1小パラグラフ\null}
\def\pageautorefname~#1\null{~#1ページ \null}
\def\appendixautorefname~#1\null{~#1 \null}
% 参照のショートカット
\newcommand\aref[1]{\autoref{#1}}

% 画像貼り付け
\usepackage{graphicx}

% その位置に指定して置く
\usepackage{float}

% タイトル・著者の使いまわし
\usepackage{authoraftertitle}

% 化学構造式(きれいに書く方)
\usepackage{xymtexps}

% 図表を次のsectionにおかないように設定
\usepackage[section]{placeins}

% 複数行の数式を書く
\usepackage{amsmath}

% ≒を使う
\usepackage{amssymb}

% セクションにドットをつける (1 見出し → 1. 見出し)
\usepackage{secdot}
\sectionpunct{section}{.}
\sectionpunct{subsection}{}
\sectionpunct{subsubsection}{}
\sectionpunct{paragraph}{}
\sectionpunct{subparagraph}{}

% 問題なくキャプションは使えるが警告が出るため黙らせる
\usepackage{silence}
\WarningFilter{caption}{Unknown document}
% キャプションを設定
\usepackage{caption}
\usepackage{varwidth}
% 中央寄せをデフォルトにしたスタイルを作成
\DeclareCaptionFormat{capfmt}{%
  % #1: label (e.g. "Table 1")
  % #2: separator (e.g. ": ")
  % #3: caption text
  \begin{varwidth}{\linewidth}%
    \centering
    {\fontcap #1#2#3}%
  \end{varwidth}%
}

% SI単位系を扱う
\usepackage{siunitx}
% \SI{数値}{単位}で表示できるが、
% {数値}にxのような変数を入れるためのコマンド\SIeqを定義
\newcommand{\SIeq}[2]{\SI[parse-numbers=false]{#1}{#2}}

% ソースコード添付用
% 日本語コメントに対応したjvlistingを使用
\usepackage{listings,jvlisting} 
\lstset{
  basicstyle={\ttfamily},
  identifierstyle={\small},
  commentstyle={\small},
  keywordstyle={\small\bfseries},
  ndkeywordstyle={\small},
  stringstyle={\small\ttfamily},
  frame={tb},
  breaklines=true,
  columns=[l]{fullflexible},
  numbers=left,
  xrightmargin=0em,
  xleftmargin=3em,
  numberstyle={\scriptsize},
  stepnumber=1,
  numbersep=1em,
  lineskip=-0.5ex
}

% 参考文献の参照を上付きに
\usepackage[super]{cite}

\begin{document}

% キャプションの設定
% 作ったスタイルを割り当て
\captionsetup{format=capfmt}
% 番号と見出しの区切りを空白に
\captionsetup{labelsep=quad}
% bold体に
\captionsetup{font=bf}

\makeatletter
    % abstractを通常のsectionと同様に変更
    \renewenvironment{abstract}{%
    \titlepage
    \@beginparpenalty\@lowpenalty\noindent
    {\Large\bfseries\abstractname}\\%
    \@endparpenalty\@M
    }%
    {\par\endtitlepage}
    \renewcommand{\abstractname}{要旨}

    % 太横線を定義
    \def\Hline{
        \noalign{\ifnum0=`}\fi\hrule \@height 4.\arrayrulewidth \futurelet
        \reserved@a\@xhline}

    % ソースコードのキャプション見出しを変更
    \renewcommand{\lstlistingname}{コード}

    % 参考文献の参照を <x)> の形式に変更
    \renewcommand\citeform[1]{#1)}

    % 参考文献のリストの見出しを <x)> の形式に変更
    \def\@biblabel#1{#1)}
\makeatother

% \\で改行。もうちょっと空白欲しいから二重に改行
% %がないと段落になって一文字下がるから注意
\title{%
    レポートタイトル1行目
    \\\\
    レポートタイトル2行目
    }

% 作者の名前を入れる
\author{S.T.}

% 表紙を作る
% もろもろは直打ち。

\begin{titlepage}

    \vspace*{60pt} \noindent
    {\LARGE 2024 年} 
    \vspace{14pt} \\
    {\LARGE example レポート} 
    \vspace{100pt} \\
    {\Large \MyTitle}
    \vspace{130pt} \\
    \begin{tabular}{ll}
        提出者     & A番 \MyAuthor  \\
        実験日     & 令和 R 年 m 月 d 日(D) \\
        提出日     & 令和 R 年 m 月 d 日(D) \\
        共同実験者 & A番 X.Y. \\
                   & A番 V.W. \\
    \end{tabular} \vspace{3pt} \\
\end{titlepage}

\begin{abstract}

ここに要旨を書く。
このページはページ番号が割り振られない。
中央寄せを解除し、他の章と同じような見た目にしている。

\end{abstract} 


\section{目的}

目的を記載する。

\section{実験器具}

\subsection{環境}

\begin{itemize}
    \item PC
    \item vscode
    \item TexLive2022
  \end{itemize}

\section{実験方法}

実験方法を記載する。

\section{実験結果}

実験結果を記載する。

\section{考察}

考察を記載する。


\section{結論}

結論を記載する。

ここに参考文献を引用する例を示す。

\cite{hon2}

\cite{hon1-a}

\cite{hon1-b}

\cite{webpage1}

\cite{handbook}


% 参考文献リストを作成

% junsrtは参照した順で単純に書くもの (japanese unsort)
\bibliographystyle{junsrt}

% simple-report.bibを文献データベースとして読み込む
\bibliography{simple-report}

% 付録を記載
\appendix

\section{付録1}

付録はappendix後に書ける

\section{付録2}

appendix後に普通にsectionを書くとA.B.に変化する


\end{document}